% !TeX root = ../Rules.tex
% !TeX spellcheck = en_US
\section{Environment}
\label{sec:environment}

\subsection{The Field}
\label{sec:field}

\subsubsection{Field Construction}

\subsubsection{Field Colors}
\label{sec:field_colors}

The colors of the soccer field are as follows:
\begin{itemize}
  \item The field (artificial turf) itself is green (color is not specified, but it should not be too dark).
  \item The lines on the field are white, whether they are taped, spray painted or made from white artificial turf.
  \item The posts and top crossbar of both goals are white.
    The net and the support structure for the net are white, gray, or black.
\end{itemize}

\subsubsection{Inside and Outside}
\subsubsection{Technical Area}

\subsubsection{Venue Setup}
\label{sec:boundaries}

Fields may be located close to one another.
Barriers will not necessarily be constructed between adjacent fields to block the robots from seeing other fields, goals, or balls.
However, barriers will be constructed to block sight between any fields that are not located at least three meters apart.
Hence, for each side of a field that is adjacent to another field, either barriers will separate the fields or at least \qty{3}{\metre} will be between the carpet of adjacent fields.

\subsubsection{Lightning Conditions}
\label{sec:lightConditions}

The lighting conditions depend on the actual venue.
Fields should be placed near or under windows where possible.
Whether window lighting is used or not, ceiling lights should be provided as necessary to ensure that most of the field is never darker than 300 Lux (400 Lux preferred).

Natural and non-natural light must be free to reach the field. The technical committee can delimit a zone near the field where humans must not stand and where any items blocking the light sources are forbidden. 

\subsection{The Ball}
\label{sec:ball}
