% !TeX root = ../Rules.tex
% !TeX spellcheck = en_US
\section{Environment}
\label{sec:environment}

\subsection{The Field}
\label{sec:field}

\subsubsection{Field Construction}
\subsubsection{Field Colors}
\subsubsection{Inside and Outside}
\paragraph{Definition of Inside and Outside}
An object (such as a robot or the ball) is considered \textit{inside} a region of the field if any part of it overlaps or touches the boundary lines that define that region, or if it is fully contained within the region. It is considered \textit{outside} the region only when no part of it remains within or on the boundary lines of that region. This definition applies to any designated area of the field, such as the field itself, penalty area, center circle, or other marked regions.

\subsubsection{Technical Area}
\subsubsection{Venue Setup}
\subsubsection{Lighting Conditions}
The \leaguenameabbr does not mandate specific or controlled lighting conditions for a match venue. It is expected that the venue provides reasonable lighting suitable for general visibility (\eg indoor with artificial lighting, outdoor with natural lighting, or a combination of both). This lighting may include variations such as glare, brightness, shadows, or mixed lighting conditions that can change throughout the match. However, the lighting must be predominantly white, and colored lighting that significantly changes the perceived color of the field or ball is not allowed. Teams participating in the \leaguenameabbr are encouraged to design their robots to handle a variety of typical lighting environments that may be encountered during a match. \info{to be discussed, add more details if needed}

\subsection{The Ball}
\label{sec:ball}
