% !TeX root = ../Rules.tex
% !TeX spellcheck = en_US
\section{Environment}
\label{sec:environment}

\subsection{The Field}
\label{sec:field}

\subsubsection{Field Construction}

\subsubsection{Field Colors}
\label{sec:field_colors}

The colors of the soccer field are as follows:
\begin{itemize}
  \item The field (artificial turf) itself is green (color is not specified, but it should not be too dark).
  \item The lines on the field are white, whether they are taped, spray painted or made from white artificial turf.
  \item The posts and top crossbar of both goals are white.
    The net and the support structure for the net are white, gray, or black.
\end{itemize}

\subsubsection{Inside and Outside}
\paragraph{Definition of Inside and Outside}
An object (such as a robot or the ball) is considered \textit{inside} a region of the field if any part of it overlaps or touches the boundary lines that define that region, or if it is fully contained within the region. It is considered \textit{outside} the region only when no part of it remains within or on the boundary lines of that region. This definition applies to any designated area of the field (See Figure~\ref{fig:field_regions}).
\paragraph{Regions of the Field}
The field is divided into several key regions. These regions include:
\begin{itemize}
    \item \textbf{Center Circle:} The circular area at the center of the field, used for kick-offs and other specific plays.
    \item \textbf{Penalty Area:} The rectangular area in front of each goal, where specific rules apply regarding fouls and goalkeeping.
    \item \textbf{Goal Area:} The smaller rectangular area within the penalty area, from which goal kicks are taken.
    \item \textbf{Boundary Lines:} The lines marking the sides of the field, where the ball is considered out of play if it crosses these lines.
    \item \textbf{Goal Lines:} The lines marking the ends of the field, where goals are scored if the ball crosses these lines between the goalposts and beneath the crossbar.
    \item \textbf{A Team's Half:} The half of the field that is closest to a team's own goal, where they primarily defend against opposing attacks.
\end{itemize}

\begin{figure}[h]
    \centering
    % Placeholder figure for annotated field diagram
    \fbox{\parbox{0.4\linewidth}{\centering Placeholder annotated field diagram}}
    \caption{Field diagram highlighting key regions.}
    \label{fig:field_regions}
\end{figure}
\unsure{Should this be included?}

\subsubsection{Technical Area}

\subsubsection{Venue Setup}
\label{sec:boundaries}

Fields may be located close to one another.
Barriers will not necessarily be constructed between adjacent fields to block the robots from seeing other fields, goals, or balls.
However, barriers will be constructed to block sight between any fields that are not located at least three meters apart.
Hence, for each side of a field that is adjacent to another field, either barriers will separate the fields or at least \qty{3}{\metre} will be between the carpet of adjacent fields.

\subsubsection{Lightning Conditions}
\label{sec:lightConditions}

The \leaguenameabbr does not mandate specific or controlled lighting conditions for a match venue. It is expected that the venue provides reasonable lighting suitable for general visibility (\eg indoor with artificial lighting, outdoor with natural lighting, or a combination of both). The lighting conditions depend on the actual venue. Fields should be placed near or under windows where possible. Whether window lighting is used or not, ceiling lights should be provided as necessary so that most of the field is at least \qty{300}{\lux} (preferably \qty{400}{\lux}). This lighting may include variations such as glare, brightness, shadows, or mixed lighting conditions that can change throughout the match. However, the lighting must be predominantly white, and colored lighting that significantly changes the perceived color of the field or ball is not allowed. Teams participating in the \leaguenameabbr are encouraged to design their robots to handle a variety of typical lighting environments that may be encountered during a match. Natural and non-natural light must be free to reach the field. The technical committee can delimit a zone near the field where humans must not stand and where any items blocking the light sources are forbidden.

\subsection{The Ball}
\label{sec:ball}
