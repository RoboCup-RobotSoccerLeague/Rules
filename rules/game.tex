% !TeX root = ../Rules.tex
% !TeX spellcheck = en_US
\section{Game Process}
\label{sec:game}

\subsection{Structure of the Game}
\label{sec:game_structure}

A game consists of four parts, pre-game meeting (see~\cref{sec:referee_team_communication}), the first half, a half-time break, and the second half.
Each half is \qty{10}{\minute} counted from the initial kick-off.
The half-time break is \qty{10}{\minute}, and during this time both teams may change robots, change programs, or do anything else that can be done within the time allotted.

The head referee signals the commencement of each half with a single whistle blow (that is, the Initial kick-off, \cf \cref{sec:initial-kick-off}).
The head referee signals the end of the first half with two short whistle blows, and the end of the second half with two short plus one long whistle blow.
The head referee should make \textit{all} of these whistle sounds behind the T-junction of the halfway line.

\subsection{Robot States}
\label{sec:robot_states}

Robots can be in \textit{ten} different \emph{primary} states (see~\cref{fig:robot_states}).
Wireless connection must be available, so these states will be set by the GameController.
Teams must implement code to receive and correctly respond to wireless GameController packets, and also give a visual indication of the game state.

Should, on both teams, at least two robots have problems with the wireless network or GameController connection the head referee should issue a referee timeout (see \cref{sec:referee_timeout}).
If fewer robots do not respond to the GameController then they are, at the beginning, not included in the game (via a `Request for Pick-up', see \cref{sec:request_for_pickup}), and the game starts without the offending robot.
The same applies to robots that no longer respond to the GameController at a later point in the game.
Not responding means in this case that they are failing to react to the next GameController signal, such as a penalty or a change in game state.

The primary states are:
\begin{description}
  \item[Unstiff.] This state ensures consistent and safe handling of robots.
    At any time, teams must have a way to manually trigger this state.
    The robot must move to a safe position and unstiffen all joints.
    So while in the \texttt{unstiff} state the robot is not allowed to move in any fashion! 
    After booting, the robots are in their \texttt{unstiff} state.
    Executing the same input while in the \texttt{unstiff} state, permits the robot to stiffen its joints and return to the \texttt{initial} state, or a state as indicated by GameController.

    \item[Initial.] The robots are free to move at teams convenience and humans are allowed to interact with the robots.
    This state is not limited in time and teams have access to the field.
    The GameController will activate this state before \texttt{ready} (i.e at the beginning of a half and during a timeout).
    A manual input can transition the robot to \texttt{unattended} when the robot is left unattended on the field.

    \item[Unattended.]
    This state is used when a human  enters or is near the field.
    The GameController can activate this state from any other state beside \texttt{playing}. 
    Robots in this state must immediately stop any motion, assume a safe posture (e.g. seated or crouched), and remain stationary.
    Robots are not allowed to resume play until the human has exited the field area and the GameController authorizes a transition to a different state.
    Teams must implement a safe and reliable method for triggering this state, both automatically and manually.

  \item[Ready.] In this state, the robots walk to their legal positions for kick-off (\cf \cref{sec:kick-off}) or a penalty kick (\cf \cref{sec:penalty_kick}).
    They remain in this state, until the head referee decides that there is no significant progress, up to a maximum of \qty{\KickOffAutoTime}{\second} for a kick-off and \qty{\PenaltyKickSetupTime}{\second} for a penalty kick.
    The GameController can activate sub-states for kick-off and penalty kicks.

  \item[Set.] In this state, the robots stop and wait for kick-off (\cf \cref{sec:kick-off}) or a penalty kick (\cf \cref{sec:penalty_kick}).
    Illegally positioned robots are \texttt{penalized} and placed on the side of the field.
    Robots are allowed to move their heads and arms or get up if fallen before the game (re)starts, but they are not otherwise allowed to move their legs or locomote in any fashion.
    If a robot cannot get up, fallen robot is called~(\cf \cref{sec:fallenrobots}).
    The penalty time counter is frozen during this state.
    Note that all penalized robots are left in place (on the side of the field, or in-place for motion in set) and must wait to get unpenalized.
    The GameController can activate sub-states for kick-off and penalty kicks.

  \item[Playing.] In the \texttt{playing} state, the robots are playing soccer.
    During the \texttt{playing} state, the GameController can activate the sub-states for free kicks (\cf \cref{sec:free_kick}).

  \item[Penalized.] A robot is in this state when it has been \texttt{penalized}.
    The robot must walk to the side of the field according to penalty procedure (\cf \cref{sec:penalty_procedure}).

  \item[Expulsed.] A robot is in this state when it has been \texttt{expulsed}.
    The only permitted movement is transitioning into a safe posture (e.g., seated or crouched).
    If the robot has fallen, it must remain down.

  \item[Pickup.] A robot enters this state when its team requests a pick-up.
    It must walk to its designated pick-up location (see \cref{sec:pickuprobots}).

  \item[Finished.] This state is reached when a half is finished.

\end{description}

The referee will announce the start of the \texttt{playing} state with a single whistle blow.
The GameController \texttt{playing} signal will be delayed by \qty{\PlayingDelayTime}{\second}.
This delay applies to both kick-off and penalty kicks.

The current game state must be displayed on the LED of the robot.
The colors corresponding to the game states are:
\begin{table}[h]
\centering
\begin{tabular}{ll}
\toprule
\textbf{Game State} & \textbf{LED Color Pattern} \\
\midrule
\texttt{Unstiff} & Blue (blinking) \\
\texttt{Initial} & White \\
\texttt{Unattended} & Alternating Cyan/Red (every 254 ms) \\
\texttt{Ready} & Blue \\
\texttt{Set} & Yellow \\
\texttt{Playing} & Green \\
\texttt{Penalized} & Red \\
\texttt{Expulsed} & Purple \\
\texttt{Pickup} & Orange \\
\texttt{Finished} & White \\
\bottomrule
\end{tabular}
\caption{LED color patterns for robot game states.}
\end{table}

The current GameController requires robots to know both their team number and their robot number within the team.
It is each team's responsibility to make sure this is correctly configured.
It is recommended that the robot indicates its number within the team on boot up so that this can be easily checked at the start of the game.

\subsection{Kick-off}
\subsubsection{Field-Side Selection and Initial Kick-off}
\subsubsection{Initial Kick-off}
\subsubsection{Kick-off}
\label{sec:kick-off}
\subsubsection{Ball in play}

\subsection{Goals}
\subsubsection{Goal Scored}
\label{sec:goal}

A goal, including own goal, is achieved when the entire ball (not only the center of the ball) goes over the goal-side edge of the goal line, \ie the ball is completely inside the goal.\footnote{
  The goal line is part of the field.
}

The head referee signals a goal by a single whistle blow, followed by the call ``Goal \textless color\textgreater''.
The head referee should point with one arm towards the center of the field.
To assist robots listening for whistles, the referee should blow the whistle from on the carpet at the end of the fields where the goal was scored.

After a team scores a goal, the game proceeds with a kick-off (\cf~\cref{sec:kick-off}) for their opponents.
The GameController signal (to the robots) of a goal being scored, will be delayed by \qty{\GoalScoredDelay}{\second}.

\subsubsection{Invalid Goal}
\subsubsection{Competition Rules}

\subsection{Kick-in / Throw-in}

\subsection{Goal-Kick}

\subsection{Corner-Kick}

\subsection{Free Kick}
\subsubsection{Direct Free Kick}
\subsubsection{Indirect Free Kick}
\subsubsection{Visual gesture}
\subsubsection{Execution}

\subsection{Indirect Kick}
\subsubsection{Fallback mode}

\subsection{Penalty Kick}

\subsection{Game Stuck}
\subsubsection{Local Game Stuck}
\subsubsection{Global Game Stuck}

\subsection{Request for Pick-up}

\subsection{Timeout}
\subsubsection{Request for Timeout}
\label{sec:request_for_timeout}

Each team can call a \textbf{maximum of 1 timeout per game} with a total time of no more than \textbf{5 minutes}.
During this time, both teams may change robots, change programs, or anything else that can be done within the time allotted.
During normal game time, a team may call a timeout at any stoppage of play (after a goal, stuck game, before a half, etc.).
Alternatively, a team may call a timeout before a penalty shootout if they have not used their timeout yet (\cf \cref{sec:penalty_shoot-out}).

The timeout ends when the team that called the timeout says they are finished, at which time they must be ready to play.
The other team must be ready to play at the time the timeout runs out, or \textbf{2 minutes} after a prematurely called end of the timeout, whichever is earlier.
If the other team is not ready to play in time, it has to call a timeout of its own.

The clock stops during timeouts, even during the preliminaries, and is reset to the time when the current stoppage of play began.

After the completion of the timeout, the game resumes with a kick-off for the team which did not call the timeout.

If a team is not ready to play at the assigned time for a game, the referee will call the timeout for that team.
After the expiration of such a timeout, if the team is still not ready to play then the referee shall start the game with only one team on the field.
The team that was not ready can return its robots to the field as per the rules for ``Request for Pick-up''.
If both teams are not ready, the referee will call timeouts for both teams.
This ``double timeout'' expires after 10 minutes.

\subsubsection{Referee Timeout}

\subsection{Extra Time}

\subsection{Mercy Rule}
\label{sec:mercy_rule}

A game will conclude once the game score shows a goal difference of 10.
Ending the game is mandatory once a goal difference of 10 is reached.

\subsection{Drop Ball Rule}

\subsection{Ball Stop Rule}

\subsection{Determine the Winner of a Match}
\subsubsection{Winning Team}
\subsubsection{Winner after Drawn}

\subsection{Penalty Kick Shoot-Out}
\subsubsection{Penalty Kick}
\subsubsection{Sudden Death Shoot-Out}
