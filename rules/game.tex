% !TeX root = ../Rules.tex
% !TeX spellcheck = en_US
\section{Game Process}
\label{sec:game}

\subsection{Structure of the Game}
\label{sec:game_structure}

\subsection{Game Periods and Robot States}
\label{sec:game_states}
\subsubsection{Game Periods}
\subsubsection{Robot States}

\subsection{Kick-off}
\subsubsection{Field-Side Selection and Initial Kick-off}
\subsubsection{Initial Kick-off}
\subsubsection{Kick-off}
\label{sec:kick-off}
\subsubsection{Ball in play}

\subsection{Goals}
\subsubsection{Goal Scored}
\label{sec:goal}

A goal, including own goal, is achieved when the entire ball (not only the center of the ball) goes over the goal-side edge of the goal line, \ie the ball is completely inside the goal.\footnote{
  The goal line is part of the field.
}

The head referee signals a goal by a single whistle blow, followed by the call ``Goal \textless color\textgreater''.
The head referee should point with one arm towards the center of the field.
To assist robots listening for whistles, the referee should blow the whistle from on the carpet at the end of the fields where the goal was scored.

After a team scores a goal, the game proceeds with a kick-off (\cf~\cref{sec:kick-off}) for their opponents.
The GameController signal (to the robots) of a goal being scored, will be delayed by \qty{\GoalScoredDelay}{\second}.

\subsubsection{Invalid Goal}
\subsubsection{Competition Rules}

\subsection{Kick-in / Throw-in}

\subsection{Goal-Kick}

\subsection{Corner-Kick}

\subsection{Free Kick}
\subsubsection{Direct Free Kick}
\subsubsection{Indirect Free Kick}
\subsubsection{Visual gesture}
\subsubsection{Execution}

\subsection{Indirect Kick}
\subsubsection{Fallback mode}

\subsection{Penalty Kick}

\subsection{Game Stuck}
\subsubsection{Local Game Stuck}
\subsubsection{Global Game Stuck}

\subsection{Request for Pick-up}

\subsection{Timeout}
\subsubsection{Request for Timeout}
\label{sec:request_for_timeout}

Each team can call a \textbf{maximum of 1 timeout per game} with a total time of no more than \textbf{5 minutes}.
During this time, both teams may change robots, change programs, or anything else that can be done within the time allotted.
During normal game time, a team may call a timeout at any stoppage of play (after a goal, stuck game, before a half, etc.).
Alternatively, a team may call a timeout before a penalty shootout if they have not used their timeout yet (\cf \cref{sec:penalty_shoot-out}).

The timeout ends when the team that called the timeout says they are finished, at which time they must be ready to play.
The other team must be ready to play at the time the timeout runs out, or \textbf{2 minutes} after a prematurely called end of the timeout, whichever is earlier.
If the other team is not ready to play in time, it has to call a timeout of its own.

The clock stops during timeouts, even during the preliminaries, and is reset to the time when the current stoppage of play began.

After the completion of the timeout, the game resumes with a kick-off for the team which did not call the timeout.

If a team is not ready to play at the assigned time for a game, the referee will call the timeout for that team.
After the expiration of such a timeout, if the team is still not ready to play then the referee shall start the game with only one team on the field.
The team that was not ready can return its robots to the field as per the rules for ``Request for Pick-up''.
If both teams are not ready, the referee will call timeouts for both teams.
This ``double timeout'' expires after 10 minutes.

\subsubsection{Referee Timeout}

\subsection{Extra Time}

\subsection{Mercy Rule}
\label{sec:mercy_rule}

A game will conclude once the game score shows a goal difference of 10.
Ending the game is mandatory once a goal difference of 10 is reached.

\subsection{Drop Ball Rule}

\subsection{Ball Stop Rule}

\subsection{Determine the Winner of a Match}
\subsubsection{Winning Team}
\subsubsection{Winner after Drawn}

\subsection{Penalty Kick Shoot-Out}
\subsubsection{Penalty Kick}
\subsubsection{Sudden Death Shoot-Out}
