% !TeX root = ../Rules.tex
% !TeX spellcheck = en_US

\section{Competition Structure}
\label{sec:competition}

The general aims of the structure are:
\begin{enumerate}
    \item \emph{Encourage} overall progress towards the long-term goal -- final game 2050;
    \item \emph{Support} fair competition, development of the new teams, and novel approaches in research and technology;
\end{enumerate}

The competition within the \leaguename is structured in three levels:
\begin{enumerate}
    \item Main Competition
    \item Mixed / Cross Competition
    \item Technical and Scientific Challenges
\end{enumerate}


Basic definitions:

\begin{description}
    \item[Team] -- a team of robots playing in a single division
        - can have several institutions working on it (joint team)
        - can have different types of robots (all need to fit the division)
    \item[Super-Team] -- a team composed of robots from different teams
        (can cross divisions)
        - robots from different teams play as a single super team 
        - these play with more robots or a larger field 
    \item[Mixed-Team] -- a team of robots with robots belonging to different divisions 
\end{description}




\subsection{Main Competition -- Divisions}
\label{sec:com_divisions}

The main competition is structured into three divisions.
The robots and teams must fulfill the following criteria to compete in the games in a particular division:

\begin{tblr}{
  width=1.0\linewidth,
  colspec = {Q[l]X[l]X[l]},
  row{1}={fg=black, font=\bfseries}, %, 
  hline{2} = {solid},
  hline{3-Y} = {dashed},
  vline{2,3} = {abovepos = -1, belowpos = -1},
}
Division & Max Robot Height & Max Robot Weight \\

small & 110 cm & 15 kg \\
mid   & 125 cm & 25 kg \\
large & any size & any weight \\
\end{tblr}

\textbf{NOTE:} The human team must be able to handle their robots safely. Robots that cannot be safely handled by the human team, for instance, because of their weight, size, or strength, are not permitted to compete. \todo{set a reference to appropriate sections in robot players and safety sections}.

\textbf{NOTE:} Robots from smaller divisions are permitted to participate in larger divisions, provided reasonable technical feasibility and safety are ensured.

Examples of robots in each division:

\begin{tblr}{
  width=1.0\linewidth,
  colspec = {Q[l]X[l]},
  row{1}={fg=black, font=\bfseries}, %, 
  hline{2} = {solid},
  hline{3-Y} = {dashed},
  vline{2} = {abovepos = -1, belowpos = -1},
}
Division & Permitted Robots \\

small & NAO, Robotis OP3 \\
mid   & Booster K1, Hightorque Hi, Hightorque Pi+ \\
large & Unitree, Booster T1, Alice 4 \\
\end{tblr}

The games \emph{within} each division are structured in two complexity levels:
\begin{description}
    \item[Foundation] -- fundamental games with minimal requirements and complexity to allow teams to develop, and to study technical and scientific foundations;
    \item[Advanced] -- game with higher complexity, pushing the boundaries of the state-of-the-art;
\end{description}

The following limits apply to foundation and advanced games in each division:

\begin{tblr}{
  width=1.0\linewidth,
  colspec = {Q[l]X[l]X[l]},
  row{1}={fg=black, font=\bfseries}, %, 
  hline{2} = {solid},
  hline{3-Y} = {dashed},
  vline{2,3} = {abovepos = -1, belowpos = -1},
}
Division & Foundation Players per Team & Advanced Players per Team\\

small & 4 & 7 \\
mid   & 3 & 5 \\
large & 3 & 5 \\
\end{tblr}


Field sizes:

\begin{tblr}{
  width=1.0\linewidth,
  colspec = {Q[l]X[c]X[c]},
  row{1}={fg=black, font=\bfseries}, %, 
  hline{2} = {solid},
  hline{3-Y} = {dashed},
  vline{2} = {abovepos = -1, belowpos = -1},
}
Division & Min Field Size in m & Max Field Size in m\\
%
small & \SetCell[c=2]{c} 6 x 9 &  \\
mid   & 6 x 9   &  9 x 14 & \\
large &  9 x 14 &  14 x 22   \\
\end{tblr}

\textbf{NOTE:} The concrete field size for the large division will be determined by the organizing committee and the local organizing committee, depending on the availability of space and resources.
\textbf{NOTE:} large field 


Ball types and sizes:

\begin{tblr}{
  width=1.0\linewidth,
  colspec = {Q[l]X[l]},
  row{1}={fg=black, font=\bfseries}, %, 
  hline{2} = {solid},
  hline{3-Y} = {dashed},
  vline{2} = {abovepos = -1, belowpos = -1},
}
Division & Allowed ball types and size\\

small & SPL ball (10 cm), FIFA mini ball size 1(14 cm) \\
mid   & FIFA size 3 (preferred), FIFA size 4 \\
large & FIFA size 5 \\
\end{tblr}

\subsection{[In Discussion] Mixed Competition -- Cross Division}
\label{sec:com_mixed}

With the future progress of the \leaguenameabbr, the robot players are expected to grow in size and weight, the team sizes get larger, and the games become more complex.

The mixed competitions aim to explore soccer games with a larger number of players on larger fields and to foster collaboration between different teams.

In the mixed competition, Super-Teams consisting of two or more teams compete in the following three categories:

\begin{tblr}{
  width=1.0\linewidth,
  colspec = {Q[l]X[l]X[l]X[l]X[l]},
  row{1}={fg=black, font=\bfseries}, %, 
  hline{2} = {solid},
  hline{3-Y} = {dashed},
  vline{2,3} = {abovepos = -1, belowpos = -1},
}
Mixed Competition & Participating Division    & Players per Super-Team & Min Players by Team  & Field Size \\
mixed-small & small       & 11 & 3 & mid  \\
mixed-mid   & mid + large & 7  & 3 & large \\
mixed-large & large       & 5  & 2 & large \\
\end{tblr}



\subsection{Technical and Scientific Challenges}
\label{sec:com_challenges}

Technical and scientific challenges aim to explore beyond the boundaries of regular competitions and do not need to follow the rules of main or mixed competitions. 
The rules for technical and scientific challenges are formulated in a separate document.

