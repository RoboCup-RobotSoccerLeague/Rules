% !TeX root = ../SPL-Rules.tex
% !TeX spellcheck = en_US
\section{Purpose and Scope of the \leaguename}
\label{sec:introduction}

This document defines the purpose, scope, and rules of the \leaguename for the international RoboCup competition, at the forefront of research, innovation, and education in humanoid soccer across all forms.

\subsection{Vision and Mission Statement}
\label{sec:vision_and_mission}

%NB: This is the vision statment developed by the former SPL teams
The \leaguename~(\leaguenameabbr) will drive innovative research that advances the software and hardware of autonomous robots with a particular emphasis on robots deployed in real-time, dynamic, partially observable, and multi-agent environments.
The \leaguenameabbr is well suited to advance \textit{research}, \textit{development}, and \textit{education} in:
\begin{itemize}
    \item Multi-robot systems (5+ robots) requiring decentralized coordination with limited communication over noisy channels.
    \item Robots that approach or approximate Human-like capabilities.
    \item A league that promotion research, 
    \item Localization and state-estimation.
    \item Dynamic humanoid motor control.
    \item Real-time and on-board robot perception.
    \item Software engineering for autonomous robots.
    \item Hardware engineering for autonomous robots.
    \item Across topics, robot learning in all its forms.
\end{itemize}

In addition, the \leaguenameabbr aim to:
\begin{itemize}
    \item Grow the community of humanoid soccer within RoboCup.
    \item Further education in robotics and is designed such that both teams with a primary focus in research and teams with a primary focus in education are able to participate.
    \item Encourage active sharing of software and hardware designs for league-wide collaboration.
    \item Measure the capabilities of the league against the 2050 vision of RoboCup.
    \item Drive the vision and direction of the rules to encourage good quality soccer between between evenly matched teams.
\end{itemize}

\subsection{Core Vision and Requirements for Legal Standard Robot Platforms}
\info{Community comment, to be discussed: They make sense as ideals, but not as strict requirements.
Reword §1.2 so that it is clear that these are ideals and not to be confused for the technical requirements of platforms as in §4. Maybe replace mentions to "requirements"}
The \leaguenameabbr encourages and welcomes the use of standard humanoid robot platforms available within the market to advance the state of humanoid robot soccer and the vision of the \leaguenameabbr.

With the \leaguenameabbr vision in mind, the core requirements for standard humanoid platforms used within the \leaguenameabbr are platforms:
\begin{enumerate}
    \item Capable of dynamic motions such as fast walking, kicking a ball off the ground, and getting up from the ground;
    \item Capable of running state-of-the-art AI neural network models for perception, decision-making, and control;
    \item Sufficiently small and affordable that teams can fund multiple robots and travel with them to competitions;
    \item Able to be programmed at a low-level of control;
    \item Well-Documented.
\end{enumerate}
\info{3 and 5 are also hard to enforce, since terms like “sufficiently small and affordable” and “well-documented” are too subjective.}

\subsection{Core Vision and Requirements for Constructed Robots}

The \leaguenameabbr equally encourages and welcomes the use of fully custom built or modified humanoid robot platforms. To create a welcoming and fair environment for all robot platforms, the \leaguenameabbr ensures the following:
\begin{enumerate}
	\item Both store-bought and custom-built robots can participate in a fair competition without risking damage to their robots.
	\item The tournament is designed such that games are interesting for all participating teams and match-ups are fair.
	\item The tournament is designed such that all currently existing teams are able to participate.
	\item Details about hardware and software of the robots is made available to teams and organizers to ensure a fair competition and encourage scientific exchange.
	\item League resources are distributed such that both store-bought and custom-built robots equally benefit from them.
	\item Robots are designed with the goal of RoboCup in mind, thus restricting the allowed sensors where possible to humanoid sensors. Exceptions to this rule can be made if it benefits scientific progress.
\end{enumerate}
